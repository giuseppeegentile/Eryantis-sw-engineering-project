\documentclass[12pt]{article}
\usepackage[utf8]{inputenc}
\usepackage[T1]{fontenc}
\usepackage[italian]{babel}

\title{Peer-Review 1: UML}
\author{Claudio Gallinella, Paolo Fasani, Andrea Moro\\Gruppo 63}

\begin{document}

\maketitle

Valutazione del diagramma UML delle classi del gruppo 62.

\section{Lati positivi}

Analizzando l'UML del gruppo 62, tra i punti di forza da noi riscontrati vi é una buona chiarezza e leggibilità dovute all'implementazioni di molteplici sottoclassi, come ad esempio la creazione di alcune classi colore dedicate.
Successivamente abbiamo apprezzato la gestione delle isole poiché sono stati in grado di controllare la meccanica di unificazione delle isole in maniera ottimale.
Infine riteniamo che siano stati capaci di implementare in modo efficiente i vari effetti delle carte personaggio, confermando la facile lettura del diagramma.

\section{Lati negativi}

Alcune relazioni tra le classi vanno riviste e probabilmente sistemate, ad esempio la relazione di composizione tra la classe "Clouds" e "Game" risulta essere invertita.
La meccanica degli assistenti, seppur chiara, può essere compattata in una sola classe invece che utilizzarne tre.
Nella classe "Clouds" abbiamo riscontrato un errore riguardante il numero di studenti che é possibile piazzare all'interno di una nuvola, il quale dipende dalla difficoltà a cui si sta giocando (3-4).

\section{Confronto tra le architetture}

Concludendo, riteniamo che le differenze principali tra il nostro e il loro UML sono la differente gestione delle isole e della loro unificazione, l'implementazione degli effetti delle carte personaggio, la gestione dei colori, la selezione della difficolta e la determinazione del vincitore.
Riteniamo che siano stati più bravi per quanto riguarda i primi due aspetti sopra citati, mentre per i rimanenti siano valide entrambe le soluzioni adottate.

\end{document}